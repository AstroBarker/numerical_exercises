\documentclass[11pt]{book}

% add an exercise environment from here
% http://www.dickimaw-books.com/latex/novices/html/newenv.html

% include figures
\usepackage{epsfig}

% prefer PDF to PNG
\DeclareGraphicsExtensions{%
  .pdf, .png}

\graphicspath{{intro/}{finite-volume/}{advection/}{burgers/}{diffusion/}{Euler/}{incompressible/}{multigrid/}{multiphysics/}}

% AMS symbols
\usepackage{amsmath,amssymb}

% cancel
\usepackage{cancel}

% Palatino font (and math symbols) -- looks nicer than the standard
% LaTeX font
\usepackage{mathpazo}
%\usepackage{helvet}

% URLs (special font for monospace)
%\usepackage{inconsolata}
%\usepackage[T1]{fontenc}


% margins and paper size -- this needs to come before fncychap
% see http://stackoverflow.com/questions/3729099/latex-paper-size
\usepackage[top=0.75in,
            bottom=0.75in,
            inner=0.9in,
            outer=0.65in]{geometry} 
\geometry{papersize={7in,10in}}

% crop marks (for printing)
%\usepackage[
%  noinfo,
%  cam,
%  cross,                % crosses as marks
%  width=7.25in,         % the width of the galley
%  height=10.25in,        % the height of the galley
%  center                % actual page is centered on the galley
%]{crop}


% this needs to be done before the fncychap style, since that will 
% redo the chapter stuff
\usepackage{sectsty}
\allsectionsfont{\sffamily}


% chapter title styles
\usepackage[Bjornstrup]{fncychap}
\ChNumVar{\fontsize{76}{80}\usefont{OT1}{pzc}{m}{n}\selectfont}
\ChTitleVar{\raggedleft\Large\sffamily\bfseries}



% hyperlinks -- load after fncychap
\usepackage{hyperref}

% color package
\usepackage{color}
\definecolor{mygray}{gray}{0.5}


% custom hrule for title page
\newcommand{\HRule}{\rule{\linewidth}{0.125mm}}

\newcommand{\pyro}{{\sf pyro}}



% footer
%% \usepackage{fancyhdr}
%% \pagestyle{fancy}
%% \fancyfoot[LO,LE]{\footnotesize \sffamily \color{mygray} M.\ Zingale---Notes on the Euler equations}
%% \fancyfoot[RO,RE]{\footnotesize \sffamily \color{mygray} (\today)}
%% \fancyfoot[CO,CE]{\thepage}
%% \fancyhead{}
%% \renewcommand{\headrulewidth}{0.0pt}
%% \renewcommand{\footrulewidth}{0.0pt}

% don't make the chapter/section headings uppercase.  See the fancyhdr
% documentation (section 9)
\usepackage{fancyhdr}
\renewcommand{\chaptermark}[1]{%  
 \markboth{\chaptername
\ \thechapter.\ #1}{}}

\renewcommand{\sectionmark}[1]{\markright{\thesection---#1}}



% don't put a header on blank pages, see
% http://www.latex-community.org/forum/viewtopic.php?f=4&p=51559
% change ``plain'' to ``empty'' to eliminate the page number
\makeatletter
\renewcommand*\cleardoublepage{\clearpage\if@twoside
\ifodd\c@page\else
\hbox{}
\thispagestyle{empty}
\newpage
\if@twocolumn\hbox{}\newpage\fi\fi\fi}
\makeatother


% skip a bit of space between paragraphs, to enhance readability
\usepackage{parskip}


% captions
\usepackage{caption}
\renewcommand{\captionfont}{\footnotesize}
\renewcommand{\captionlabelfont}{\footnotesize}
\setlength{\captionmargin}{3em}


%-----------------------------------------------------------------------------
% define a new environment for exercises
%\newcounter{exercise}    % simple way -- no TOC list
% toclot allows us to make a list of exercises
% the new environment stuff came from http://www.dickimaw-books.com/latex/novices/html/newenv.html
\usepackage{tocloft}
\newcommand{\listexercisename}{List of Exercises}

% the [chapter] argumenet here means that we reset the numbers each chapter
% exc is the short name we give to this new list
\newlistof[chapter]{exercise}{exc}{\listexercisename}  
\usepackage{changepage}   % used to adjust the margins within the environment

% environment name, with optional argument for display in the list of
% exercises
\newenvironment{exercise}[1][]
{% begin code 
  %
  % this indents our text from the left and right
  \begin{adjustwidth}{1.0cm}{1.0cm}
  %
  % this skips a line before 
  \par\vspace{\baselineskip}\noindent 
  %
  % this increments our counter
  \refstepcounter{exercise}%
  %
  % this displays the exercise number and starts italics
  \textbf{Exercise \theexercise:}\ \begin{itshape}% 
  %
  % this adds an entry to our custom List of Exercises
  \addcontentsline{exc}{exercise}{\protect\numberline{\theexercise} #1}
}% 
{% end code -- this is done at the end of the environment
  %
  % stop the italics
  \end{itshape}%
  %
  % this stops our custom margins
  \end{adjustwidth}%
  %
  % this skips a line at the end
  \vspace{\baselineskip}\ignorespacesafterend 
}

% this makes the List of Exercises have a line break before each chapter,
% just like in the main ToC.
%
% this has a bug -- it does not omit the space if the chapter has 0 exercises
\usepackage{etoolbox}
\preto\section{%
  \ifnum\value{chapter}=1{}\else   % don't skip before Ch 1
    \ifnum\value{section}=0\addtocontents{exc}{\vskip10pt}\fi
  \fi
}

% indent these the standard amount for a chaptered section
\setlength{\cftexerciseindent}{1.5em}

%-----------------------------------------------------------------------------

% license
\usepackage{ccicons}

% computer keyboard symbol
\usepackage{marvosym}

\newcommand{\evm}{{(-)}}
\newcommand{\evz}{{(\circ)}}
\newcommand{\evp}{{(+)}}
\newcommand{\enu}{{(\nu)}}

% shortcuts
\newcommand{\Dux}{\overline{\Delta u}^{(x)}}
\newcommand{\Duy}{\overline{\Delta u}^{(y)}}

\newcommand{\Dvx}{\overline{\Delta v}^{(x)}}
\newcommand{\Dvy}{\overline{\Delta v}^{(y)}}


% for dotted lines in the matrics/arrays
\usepackage{arydshln}



% fonts for TOC, list of figures, etc
\renewcommand*\listfigurename{\bf\textsf{List of Figures}}
\renewcommand*\listexercisename{\bf\textsf{List of Exercises}}
\renewcommand*\contentsname{\bf\textsf{Table of Contents}}


% short table of contents
%\usepackage{shorttoc}


% pack more figures on a page
\usepackage{float}
\renewcommand\floatpagefraction{.9}
\renewcommand\topfraction{.9}
\renewcommand\bottomfraction{.9}
\renewcommand\textfraction{.1}   


\newcommand{\hydroex}{{\sf hydro\_examples}}
\newcommand{\hydroexdoit}[1]{{\color{red} \LARGE \Keyboard}\/ \hydroex: {\tt #1}}
\begin{document}

\frontmatter

\begin{titlepage}

\ \\[2.5in]
\begin{center}
\HRule\\[0.5em]
{\Huge \textsf{{
Lecture Notes on\\[0.1em]
Computational Hydrodynamics\\[0.3em]
for Astrophysics}}
}
\HRule
\\[2em]

{\Large \sf Michael Zingale} \\ {\sf Stony Brook University}
\end{center}

\vfill

\begin{flushright}
\today
\end{flushright}

\end{titlepage}

\null \vfill 

\noindent \ccCopy\ 2013, 2014 Michael Zingale \\
\noindent document git version: \input git_info.tex

\noindent \ccbyncnd \\
\noindent This work is licensed under the Creative Commons
Attribution-NonCommercial-NoDerivs 3.0 Unported (CC BY-NC-ND 3.0)
license

\clearpage

%\shorttoc{Chapter Listing}{0}


\setcounter{tocdepth}{2}
\tableofcontents

\clearpage

\listoffigures
\addcontentsline{toc}{chapter}{list of figures}

\clearpage

\listofexercise
\addcontentsline{toc}{chapter}{list of exercises}

\clearpage

\chapter*{preface}
\chaptermark{preface}
\addcontentsline{toc}{chapter}{preface}


These notes are intended to help new students come up to speed on the
common methods used in computational hydrodynamics for astrophysical
flows.  They are written at a level appropriate for upper-level
undergraduates.

These are very much a work in progress, likely not spell-checked, and 
definitely not complete.  These notes are updated at irregular
intervals, usually when I have a new student working with me, or if
I am using them for a course.

The best way to understand the methods described here is to run
them for yourself.  There are two sets of example codes that
go along with these notes.

\begin{enumerate}
\item \hydroex\ is a set of simple 1-d, standalone python scripts
  the illustrate some of the basic solvers.  Many of the figures
  in these notes were created using these codes---the relevant
  script will be noted in the figure caption.  

  You can get this set of scripts from github at:\\
  \url{https://github.com/zingale/hydro_examples/}

  References to the scripts in \hydroex\ are shown throughout
  the text as: \\[0.5em]
  \hydroexdoit{scriptname}

\item  The {\sf pyro} code is a 2-d program that has
  solvers for advection, diffusion, compressible and incompressible
  hydrodynamcs.  It is designed to make 
  experimentation easy.  

  You can download {\sf pyro} at: \\
  \url{https://github.com/zingale/pyro2/} 

  and more information can be found at: \\
\url{http://bender.astro.sunysb.edu/hydro_by_example/}
\end{enumerate}

These notes benefited from numerous conversations with Ann Almgren,
John Bell, Alan Calder, Chris Malone, \& Andy Nonaka.

If you find errors, please e-mail me at Michael.Zingale@stonybrook.edu

\clearpage

\mainmatter


\chapter{Simulation Overview}

\input intro/intro.tex

%\addtocontents{exc}{\protect\addvspace{10pt}}%
\chapter{Finite-Volume Grids}

\input finite-volume/finite-volume.tex

%\addtocontents{exc}{\protect\addvspace{10pt}}%
\chapter{Advection}

\input advection/advection.tex

%\addtocontents{exc}{\protect\addvspace{10pt}}%
\chapter{Burgers' Equation}

\input burgers/burgers.tex

%\addtocontents{exc}{\protect\addvspace{10pt}}%
\chapter{Euler Equations}

\input Euler/Euler.tex

%\addtocontents{exc}{\protect\addvspace{10pt}}%
\chapter{Multigrid and Elliptic Equations}

\input multigrid/multigrid.tex

%\addtocontents{exc}{\protect\addvspace{10pt}}%
\chapter{Diffusion}

\input diffusion/diffusion.tex

%\addtocontents{exc}{\protect\addvspace{10pt}}%
\chapter{Multiphysics Applications}

\input multiphysics/multiphysics.tex

%\addtocontents{exc}{\protect\addvspace{10pt}}%
\chapter{Incompressible Flow and Projection Methods}

\input incompressible/incompressible.tex

%% \appendix

%% \chapter{Using \pyro}

%% \input pyro/pyro.tex

%% \chapter{Software Engineering Practices}

%% \input software-engineering/software-engineering.tex

\backmatter

\addcontentsline{toc}{chapter}{References}

\bibliographystyle{plain}
\bibliography{refs}

\end{document}
