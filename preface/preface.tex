
This text started as notes for new students at Stony Brook University
working on projects in computational astrophysics.  They are
intended  to help new students come up to speed on the
common methods used in computational hydrodynamics for astrophysical
flows.  They are written at a level appropriate for upper-level
undergraduates.  The focus is on discussing the {\em practical issues}
that arise when implementing these methods, with less emphasis on the
underlying theory---references are provided to fill in details where
necessary.

An underlying principle is that source code is provided for all the 
methods described here, including all the figures.  This allows
the reader to explore the routines themselves.

These are very much a work in progress, and new chapters will be
added with time.  The page size is formatted for easy reading
on a tablet or for 2-up printing in a landscape orientation on
letter-sized paper.  

This text is part of the Open Astrophysics Bookshelf.
Contributions to these notes are welcomed.  The \LaTeX\ source
for these notes is available online on github at: \\[0.25em]
%
\url{https://github.com/Open-Astrophysics-Bookshelf/numerical_exercises} \\[0.25em]
%
Simply fork the notes, hack away, and submit a pull-request to add
your contributions.  All contributions will be acknowledged in the text.


A PDF version of the notes is always available
at: \\[0.25em]
%
\url{http://bender.astro.sunysb.edu/hydro_by_example/CompHydroTutorial.pdf} \\[0.25em]
%
These notes are updated at irregular intervals, usually when I have a
new student working with me, or if I am using them for a course.

The source (usually python) for all the figures is also contained in
the main git repo.  The line drawings of the grids are done using the
classes in
\href{https://github.com/Open-Astrophysics-Bookshelf/numerical_exercises/blob/master/grid_plot.py}{{\tt
    grid\_plot.py}}.  This needs to be in your {\tt PYTHONPATH} if you
wish to run the scripts.

The best way to understand the methods described here is to run
them for yourself.  There are several sets of example codes that
go along with these notes:

\begin{enumerate}
\item \hydroex\ is a set of simple 1-d, standalone python scripts
  that illustrate some of the basic solvers.  Many of the figures
  in these notes were created using these codes---the relevant
  script will be noted in the figure caption.  

  You can get this set of scripts from github at:\\
  \url{https://github.com/zingale/hydro_examples/}

  References to the scripts in \hydroex\ are shown throughout
  the text as: \\[0.5em]
  \hydroexdoit{scriptname} \\[0.5em]
  Clicking on the name of the script will bring up the source code
  to the script (on github) in your web browser.

  More details on the codes available in \hydroex\ are described
  in Appendix~\ref{app:hydroex}.

\item  
  The \pyro\ code~\cite{pyro} is a 2-d simulation code with
  solvers for advection, diffusion, compressible and incompressible
  hydrodynamics, as well as multigrid.  A gray flux-limited diffusion
  radiation hydrodynamics solver is
  in development.  \pyro\ is designed with clarity in mind and to make 
  experimentation easy.  

  You can download \pyro\ at: \\
  \url{https://github.com/zingale/pyro2/} 

  A brief overview of \pyro\ is given in Appendix~\ref{app:pyro},
  and more information can be found at: \\
\url{http://zingale.github.io/pyro2/}

\item \hydrooned\ is a simple one-dimensional compressible
  hydrodynamics code that implements the piecewise parabolic method
  from Chapter~\ref{ch:compressible}.  It can be obtained from
  github at:\\
  \url{https://github.com/zingale/hydro1d/}

  Details on it are given in
  Appendix~\ref{app:hydro1d}.
\end{enumerate}

These notes benefited {\em immensely} from numerous conversations and
an ongoing collaboration with Ann Almgren, John Bell, Andy Nonaka, \&
Weiqun Zhang---pretty much everything I know about projection methods
comes from working with them.  Discussions with Alan Calder, Sean
Couch, Max Katz, and Chris Malone have also been influential in the
presentation of these notes.

If you find errors, please e-mail me at michael.zingale@stonybrook.edu,
or issue a pull request to the git repo noted above.  



\begin{flushright}
Michael Zingale \\
Stony Brook University
\end{flushright}


\clearpage

\section*{Authorship}

\subsection*{Primary Author}

Michael Zingale (Stony Brook)


\subsection*{Contributions}

Thank you to the
following people for pointing out typos or confusing remarks in the text:
\begin{itemize}
\item Chen-Hung
\item Chris Malone 
\item Rixin Li (Arizona)
\item Zhi Li (Shanghai Astronomical Observatory)
\end{itemize}

See the git log for full details on contributions.  All contributions
via pull-requests will be acknowledged here.
