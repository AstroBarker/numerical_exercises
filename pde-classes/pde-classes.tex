Partial differential equations are usually grouped into one of 
three different classes: {\em hyperbolic}, {\em parabolic}, 
or {\em elliptic}.


\section{Hyperbolic PDEs}

The canonical hyperbolic PDE is the wave equation:
\begin{equation}
  \frac{\partial^2 \phi}{\partial t^2} = c^2 \frac{\partial^2 \phi}{\partial x^2}
\end{equation}   
The general solution to this is traveling waves in either direction:
\begin{equation}
  \phi(x,t) = \alpha f_0(x - ct) + \beta g_0(x + ct)
\end{equation}
Here $f_0$ and $g_0$ are set by the initial
conditions, and the solution propages $f_0$ to the right and $g_0$ to
the left at a speed $c$.

A system of first-order hyperbolic PDEs takes the form:
\begin{equation}
a_t + A a_x = 0
\end{equation}
where $a = (a_0, a_1, \ldots a_{N-1})^\intercal$ and $A$ is a matrix.
This system is hyperbolic is the eigenvalues of $A$ are real.

Characteristics


\section{Elliptic PDEs}

The canonical elliptic PDE is the Poisson equation:
\begin{equation}
  \nabla^2 \phi = f
\end{equation}
Note that there is no time-variable here.  This is a pure boundary
value problem.  The solution, $\phi$ is determined completely by the
source, $f$, and the boundary conditions.


\section{Parabolic PDEs}

The canonical parabolic PDE is the heat equation:
\begin{equation}
  \frac{\partial \phi}{\partial t} = k \frac{\partial^2 f}{\partial x^2}
\end{equation}
This has aspects of both hyperbolic and elliptic PDEs.



