
\section{Introduction}

Many astrophysical problems involve modeling nuclear reactions coupled
with hydrodynamics.  For multifluid flows, the Euler equations are
augments with continuity equations for each of the (chemical or
nuclear) species.  If we denote the density of species $k$ as
$\rho_k$, then conservation of mass for each species individually
implies:
\begin{equation}
\label{eq:euler:partialdensitycont}
\ddt{\rho_k} + \nabla \cdot (\rho_k \Ub) = 0
\end{equation}
Summing Eq.~\ref{eq:euler:partialdensitycont} over $k$ gives us back
the mass contunity equation, since $\sum_k \rho_k = \rho$.  The
$\rho_k$ are sometimes called {\em partial densities}.  It is often
easier to define a mass fraction of species $k$, $X_k$, as
\begin{equation}
X_k = \frac{\rho_k}{\rho}
\end{equation}
and it is easy to see that $\sum_k X_k = 1$, and 
\begin{equation}
\frac{\partial (\rho X_k)}{\partial t} + \nabla \cdot  (\rho X_k \Ub) = 0
\end{equation}
Using the continuity equation, we can write this as an advection
equation:
\begin{equation}
\frac{\partial X_k}{\partial t} + \Ub \cdot \nabla X_k = 0
\end{equation}
or 
\begin{equation}
\DDt{X_k} = 0
\end{equation}
This latter form says that as a fluid element advects with the flow,
its composition does not change.  However, reactions can turn one
species into another, so for reacting flow, $DX_k/Dt \ne 0$.  If we
define the creation rate for species $k$ as $\omegadot_k$, then we have:
\begin{equation}
\DDt{X_k} = \omegadot_k
\end{equation}
Reactions will also release (or consume) energy, due to the change in
nuclear or chemical binding energy.  This energy will be a source to
the internal energy.  In particular, if we revisit the first law of
thermodynamics for our system,\MarginPar{this assumes chemical equilibrium}
\begin{equation}
T ds = de + p d \left( \frac{1}{\rho}\right ) = q_\mathrm{react}
\end{equation}
where $q_\mathrm{react}$ is the specific energy release due to
reactions.  Following a fluid element, this takes the form:
\begin{equation}
\DDt{e} + p \DDt{(1/\rho)} = \Hnuc
\end{equation}
Where now $\Hnuc$ is the time-rate of energy release per unit mass.

The conservative Euler equations with reactive source terms
appear as:
\begin{align}
\ddt{\rho} + \nabla \cdot (\rho \Ub) &= 0 \\
\ddt{\rho X_k} + \nabla \cdot (\rho \Ub X_k) &= \rho \omegadot_k \\
\ddt{(\rho \Ub)} + \nabla \cdot (\rho \Ub \Ub) + \nabla p &= 0 \\
\ddt{(\rho E)} + \nabla \cdot (\rho E \Ub + p \Ub ) &= \rho \Hnuc
\end{align}


\section{Adding species to hydrodynamics}

When we now consider our primitive variables: $\qb = (\rho, u, p, X_k)$,
we find
\begin{equation}
\Ab(\qb) = \left ( \begin{array}{cccc} u  & \rho     & 0      &  0\\
                                  0  &  u       & 1/\rho &  0\\
                                  0  & \gamma p & u      &  0\\
                                  0  & 0        & 0      & u \end{array} \right )
\end{equation}
There are now 4 eigenvalues, with the new one also being simply $u$.
This says that the species simply advect with the flow.  The right
eigenvectors are now:
\begin{equation}
\rb^\evm = \left ( \begin{array}{c} 1 \\ -c/\rho \\ c^2 \\ 0\end{array} \right )
%
\qquad
\rb^\evz = \left ( \begin{array}{c} 1 \\ 0 \\ 0  \\ 0\end{array} \right )
%
\qquad
\rb^\evzs{X} = \left ( \begin{array}{c} 0 \\ 0  \\ 0 \\ 1 \end{array} \right )
%
\qquad
\rb^\evp = \left ( \begin{array}{c} 1 \\ c/\rho \\ c^2 \\ 0 \end{array} \right )
\end{equation}
corresponding to $\lambda^\evm = u -c$, $\lambda^\evz = u$,
$\lambda^\evzs{X} = u$, and $\lambda^\evp = u + c$.  We see that for
the species, the only non-zero element is for one of the $u$
eigenvectors ($\rb^\evzs{X}$).  This means that $X_k$ only jumps over
this middle wave.  In the Riemann solver then, there is no `star'
state for the species, it just jumps across the contact wave.

To add species into the solver, you simply need to reconstruct $X_k$
as described above, find the interface values using this new $\Ab(\qb)$
and associated eigenvectors, solve the Riemann problem, with $X_k$ on
the interface being simply the left or right state depending on the
sign of the contact wave speed, and do the conservative update for
$\rho X_k$ using the species flux.

One issue that can arise with species is that even if $\sum_k X_k = 1$
initially, after the update, that may no longer be true.  There are a
variety of ways to handle this:
\begin{itemize}
\item You can update the species, $(\rho X_k)$ to the new time and then
define the density to be $\rho = \sum_k (\rho X_k)$---this means that
you are not relying on the value of the density from the mass continuity
equation itself.

\item You can force the interface states of $X_k$ to sum to 1.  Because
the limiting is non-linear, this is where problems can arise.  If the
interface values of $X_k$ are forced to sum to 1 (by renormalizing), then
the updated cell-centered value of $X_k$ will as well.  This is the
approach discussed in \cite{plewamuller:1999}.

\item You can design the limiting procedure to preserve the summation
property.  This approach is sometimes taken in the combustion field.
For piecewise linear reconstruction, this can be obtained by computing
the limited slopes of all the species, and taking the most restrictive
slope and applying this same slope to all the species.
\end{itemize}


\section{Burning fronts}
