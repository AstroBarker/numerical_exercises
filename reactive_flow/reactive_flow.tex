
\section{Introduction}

Many astrophysical problems involve modeling nuclear reactions coupled
with hydrodynamics.  For multifluid flows, the Euler equations are
augments with continuity equations for each of the (chemical or
nuclear) species.  If we denote the density of species $k$ as
$\rho_k$, then conservation of mass for each species individually
implies:
\begin{equation}
\label{eq:euler:partialdensitycont}
\ddt{\rho_k} + \nabla \cdot (\rho_k \Ub) = 0
\end{equation}
Summing Eq.~\ref{eq:euler:partialdensitycont} over $k$ gives us back
the mass contunity equation, since $\sum_k \rho_k = \rho$.  The
$\rho_k$ are sometimes called {\em partial densities}.  It is often
easier to define a mass fraction of species $k$, $X_k$, as
\begin{equation}
X_k = \frac{\rho_k}{\rho}
\end{equation}
and it is easy to see that $\sum_k X_k = 1$, and 
\begin{equation}
\frac{\partial (\rho X_k)}{\partial t} + \nabla \cdot  (\rho X_k \Ub) = 0
\end{equation}
Using the continuity equation, we can write this as an advection
equation:
\begin{equation}
\frac{\partial X_k}{\partial t} + \Ub \cdot \nabla X_k = 0
\end{equation}
or 
\begin{equation}
\DDt{X_k} = 0
\end{equation}
This latter form says that as a fluid element advects with the flow,
its composition does not change.  However, reactions can turn one
species into another, so for reacting flow, $DX_k/Dt \ne 0$.  If we
define the creation rate for species $k$ as $\omegadot_k$, then we have:
\begin{equation}
\DDt{X_k} = \omegadot_k
\end{equation}
Reactions will also release (or consume) energy, due to the change in
nuclear or chemical binding energy.  This energy will be a source to
the internal energy.  In particular, if we revisit the first law of
thermodynamics for our system,\MarginPar{this assumes chemical equilibrium}
\begin{equation}
T ds = de + p d \left( \frac{1}{\rho}\right ) = q_\mathrm{react}
\end{equation}
where $q_\mathrm{react}$ is the specific energy release due to
reactions.  Following a fluid element, this takes the form:
\begin{equation}
T\DDt{s} = \DDt{e} + p \DDt{(1/\rho)} = \Hnuc
\end{equation}
Where now $\Hnuc$ is the time-rate of energy release per unit mass.

The conservative Euler equations with reactive source terms
appear as:
\begin{align}
\ddt{\rho} + \nabla \cdot (\rho \Ub) &= 0 \\
\ddt{\rho X_k} + \nabla \cdot (\rho \Ub X_k) &= \rho \omegadot_k \\
\ddt{(\rho \Ub)} + \nabla \cdot (\rho \Ub \Ub) + \nabla p &= 0 \\
\ddt{(\rho E)} + \nabla \cdot (\rho E \Ub + p \Ub ) &= \rho \Hnuc
\end{align}

In primitive form, the derivation of the pressure equation is a
complex.  Previously, we found the pressure evolution in the absence
of sources by differentiating pressure along streamlines
(Eq.~\ref{eq:euler:presnosource}):
\begin{equation}
\frac{Dp}{Dt} = \left . \frac{\partial p}{\partial \rho} \right |_s
     \DDt{\rho} + 
     \left . \frac{\partial p}{\partial s} \right |_\rho
     \DDt{s}
\end{equation}
where, as before, we recognize that $\Gamma_1 \equiv \partial \log p/\partial \log \rho |_s$.
Now, however, we have an entropy source, so $Ds/Dt \ne 0$.  
The Maxwell relations\footnote{see, for instance, \cite{shu}, for some discussion
on these relations} tell us that
\begin{equation}
\partial p/\partial s |_\rho = \rho^2 \partial T/\partial \rho |_s
\end{equation}
This gives us the form:
\begin{equation}
\frac{Dp}{Dt} = \frac{p}{\rho} \Gamma_1 \frac{D\rho}{Dt} +
     \left . \rho^2 \frac{\partial T}{\partial \rho} \right |_s
     \frac{\Hnuc}{T}
\end{equation}

We need an expression for $\partial T/\partial \rho |_s$ in terms of
derivatives of $\rho$ and $T$ (since that is what our equations of
state typically provide).  Consider $\Gamma_1$, with $p = p(\rho, T)$:
\begin{equation}
  \Gamma_1 = \frac{\rho}{p} \left . \frac{dp}{d\rho} \right |_s
  = \frac{\rho}{p} \left [ p_\rho + p_T \left . \frac{dT}{d\rho} \right |_s \right ]
\end{equation}
where we use the shorthand $p_\rho \equiv \partial p/\partial \rho |_T$ and
$p_T \equiv \partial p/\partial T |_\rho$.  This tells us that
\begin{equation}
  \left . \frac{\partial T}{\partial \rho} \right |_s = \frac{p}{\rho p_T} (\Gamma_1 - \chi_\rho)
\end{equation}
where $\chi_\rho \equiv \partial \log p / \partial \log \rho |_T$.  To further simplify this,
we use some standard relations for general equations of state
\begin{equation}
\frac{\Gamma_1}{\chi_\rho} = \frac{c_p}{c_v}
\end{equation}
and
\begin{equation}
  c_p - c_v = \frac{p}{\rho T} \frac{\chi_T^2}{\chi_\rho}
\end{equation}
with $\chi_T \equiv \partial \log p / \partial \log T |_\rho$
(see, e.g., \cite{HKT} for both of these relations).  These allow us to write
\begin{equation}
  \left . \frac{\partial T}{\partial \rho} \right |_s = \frac{p\chi_T}{\rho^2 c_v}
\end{equation}

Putting this all together, we have:
\begin{equation}
\frac{Dp}{Dt} = \frac{p}{\rho}\Gamma_1  \frac{D\rho}{Dt}
   + \frac{p \chi_T}{c_v T} \dot{q}
\end{equation}
and using the continuity equation, $D\rho/Dt = -\rho \nabla \cdot \Ub$, 
we finally have:
\begin{equation}
\label{eq:reacting:p}
\ddt{p} + \Ub \cdot \nabla p + \Gamma_1 p \nabla \cdot \Ub = \Gamma_1 p \sigma \Hnuc
\end{equation}
where we defined
\begin{equation}
\sigma = \frac{\partial p/\partial T |_\rho}{\rho c_p \partial p/\partial \rho |_T}
\end{equation}
The $\sigma$ notation follows from \cite{ABRZ:II}.

For simplicity of notation, we'll write things out in terms of a
one-dimensional system for the following.  The primitive variable
system, in one-dimension, is then
\begin{align}
\ddt{\rho} + u \ddx{\rho} + \rho \ddx{u} &= 0\\
\ddt{u} + u \ddx{u} + \frac{1}{\rho} \ddx{p} &= 0\\
\ddt{p} + u \ddx{p} + \Gamma_1 p \ddx{u} &= \Gamma_1 p \sigma \Hnuc \\
\ddt{X_k} + u \ddx{X_k} &= \omegadot_k 
\end{align}


\section{Operator splitting approach}

Using operator splitting, we separate the equations into hydrodynamics
and reactive parts and we do this operations in turn.  Following the
ideas from \S~\ref{sec:multiphys:diffreact}, we perform the update
in stages.  If we denote the reaction update operator as $R_{\Delta t}$
and the hydrodynamics operator as $A_{\Delta t}$, then our update 
appears as:
\begin{equation}
  \Uc^{n+1} = R_{\Delta t/2} A_{\Delta t} R_{\Delta t/2} \Uc^n
\end{equation}

To make it more explicit, we write out system as:
\begin{equation}
\ddt{\Uc} + \nabla \cdot \Fb(\Uc) = \Rc(\Uc)
\end{equation}
where $\Rc = (0, \omegadot_k, 0, \rho \Hnuc)^\intercal$ is the vector
of reactive sources.  The advance through $\Delta t$ in time is:
\begin{enumerate}
\item {\em Evolve reaction part for $\Delta t/2$}.

  We define $\Uc_i^\star$ as the result of integrating
  \begin{equation}
    \frac{d\Uc_i}{dt} = \Rc(\Uc_i)
  \end{equation}
  over $t \in [t^n, t^{n+\myhalf}]$ with initial conditions, 
  $\Uc_i(t^n) = \Uc_i^n$.

\item {\em Solve the hydrodynamics portion for $\Delta t$, beginning
    with the state $\Uc^\star$}
   %
   \begin{equation}
     \frac{\Uc^{\star\star}_i - \Uc^\star_i}{\Delta t} =
      \frac{\Fb^\exx(\Uc^{\star,n+\myhalf}_{i-\myhalf}) - \Fb^\exx(\Uc^{\star,n+\myhalf}_{i+\myhalf})}{\Delta x}
   \end{equation}
  where $\Uc^{\star,n+\myhalf}_{i-\myhalf}$ is the predict interface state
  (using the ideas from \S~\ref{sec:onedrecon}), beginning with the state
  $\Uc^\star$.

\item {\em Evolve reaction part for $\Delta t/2$}

  We reach the final update, $\Uc^{n+1}$, by integrating
  \begin{equation}
    \frac{d\Uc}{dt} = \Rc(\Uc)
  \end{equation}
  over $t \in [t^{\myhalf}, t^{n+1}]$ with initial conditions
    $\Uc(t^{n+\myhalf}) = \Uc^{\star\star}$.

\end{enumerate}



\subsection{Adding species to hydrodynamics}

In the hydrodynamics algorithm, we neglect the reaction terms.  Their effects
are incorporated implicitly by performing the hydrodynamics update off of the 
state that has already experienced burning.
\begin{align}
\ddt{\rho} + \nabla \cdot (\rho \Ub) &= 0 \\
\ddt{(\rho X_k)} + \nabla \cdot (\rho \Ub X_k) &= 0 \\
\ddt{(\rho \Ub)} + \nabla \cdot (\rho \Ub \Ub) + \nabla p &= 0 \\
\ddt{(\rho E)} + \nabla \cdot (\rho E \Ub + p \Ub ) &= 0
\end{align}



When we now consider our primitive variables: $\qb = (\rho, u, p, X_k)$,
we find
\begin{equation}
\Ab(\qb) = \left ( \begin{array}{cccc} u  & \rho     & 0      &  0\\
                                  0  &  u       & 1/\rho &  0\\
                                  0  & \Gamma_1 p & u      &  0\\
                                  0  & 0        & 0      & u \end{array} \right )
\end{equation}
There are now 4 eigenvalues, with the new one also being simply $u$.
This says that the species simply advect with the flow.  The right
eigenvectors are now:
\begin{equation}
\rb^\evm = \left ( \begin{array}{c} 1 \\ -c/\rho \\ c^2 \\ 0\end{array} \right )
%
\qquad
\rb^\evz = \left ( \begin{array}{c} 1 \\ 0 \\ 0  \\ 0\end{array} \right )
%
\qquad
\rb^\evzs{X} = \left ( \begin{array}{c} 0 \\ 0  \\ 0 \\ 1 \end{array} \right )
%
\qquad
\rb^\evp = \left ( \begin{array}{c} 1 \\ c/\rho \\ c^2 \\ 0 \end{array} \right )
\end{equation}
corresponding to $\lambda^\evm = u -c$, $\lambda^\evz = u$,
$\lambda^\evzs{X} = u$, and $\lambda^\evp = u + c$.  We see that for
the species, the only non-zero element is for one of the $u$
eigenvectors ($\rb^\evzs{X}$).  This means that $X_k$ only jumps over
this middle wave.  In the Riemann solver then, there is no `star'
state for the species, it just jumps across the contact wave.

To add species into the solver, you simply need to reconstruct $X_k$
as described above, find the interface values using this new $\Ab(\qb)$
and associated eigenvectors, solve the Riemann problem, with $X_k$ on
the interface being simply the left or right state depending on the
sign of the contact wave speed, and do the conservative update for
$\rho X_k$ using the species flux.


One issue that can arise with species is that even if $\sum_k X_k = 1$
initially, after the update, that may no longer be true.  There are a
variety of ways to handle this:
\begin{itemize}
\item You can update the species, $(\rho X_k)$ to the new time and then
define the density to be $\rho = \sum_k (\rho X_k)$---this means that
you are not relying on the value of the density from the mass continuity
equation itself.

\item You can force the interface states of $X_k$ to sum to 1.  Because
the limiting is non-linear, this is where problems can arise.  If the
interface values of $X_k$ are forced to sum to 1 (by renormalizing), then
the updated cell-centered value of $X_k$ will as well.  This is the
approach discussed in \cite{plewamuller:1999}.

\item You can design the limiting procedure to preserve the summation
property.  This approach is sometimes taken in the combustion field.
For piecewise linear reconstruction, this can be obtained by computing
the limited slopes of all the species, and taking the most restrictive
slope and applying this same slope to all the species.
\end{itemize}


\subsection{Integrating the reaction network}

The reactive portion of our system is:
\begin{align}
\frac{dX_k}{dt} &= \omegadot_k \\
\frac{d(\rho e)}{dt} &= \rho \Hnuc
\end{align}
As written, this system is not closed without the equation of state,
since $\omegadot_k = \omegadot_k(\rho, T, X_k)$, and likewise for
$\Hnuc$.  We can express the internal energy in terms of the
temperature as
\if DEBUG
\footnote{formally, there should be a composition term here too, of the
form
\[
\sum_k \left . \frac{\partial e}{\partial X_k} \right |_{\rho,T} \frac{dX_k}{dt}  =
      \sum_k \mu_k \frac{dX_k}{dt} 
\]
where $\mu_k$ is the chemical potential.  But this term is zero if we are in chemical
equilibrium, which is usually assumed (but perhaps should be checked) for astrophysical
flows.  
}
\fi
\begin{align}
\frac{de}{dt} &= \left . \frac{\partial e}{\partial T}\right |_\rho \frac{dT}{dt} +
                 \left . \frac{\partial e}{\partial \rho}\right |_T \frac{d\rho}{dt} \\
              &= c_v \frac{dT}{dt} + \cancelto{0}{\left . \frac{\partial e}{\partial \rho}\right |_T \frac{d\rho}{dt}}
\end{align}
where the latter term cancels only when we use operator splitting,
because we do not evolve the density during the reaction step.  Note,
we'd get a slightly different temperature evolution equation if we
started with enthalpy, $h$, instead of internal energy, with $c_p$
replacing $c_v$.  Again, this difference only arises because we are
neglecting the hydrodynamic portions of the temperature evolution
during the reaction step.

Many simulation codes neglect the evolution of temperature during the
reaction step (see, e.g., \cite{flash}).

An excellent introduction to integrating astrophysical nuclear
reaction networks is found in \cite{timmes_nets}.  The main issue with
reaction networks is that they tend to be stiff, which means that
implicit integration techniques are needed.  There are a number of
well-tested, freely available implicit ODE integrators
(e.g.~\cite{vode}).

For stiff systems, you need to supply both a righthand side routine
and a Jacobian routine\footnote{although this could be calculated via
  finite-differences if needed}.



\section{Burning modes}

\subsection{Convective burning}


\subsection{Deflagrations}


\subsection{Detonations}

