

\begin{quote}
\noindent {\em Here we describe the basic structure and use of the
pyro code that implements many of the algorithms described in these
notes.}
\end{quote}


\section{Getting \pyro}

\pyro\ can be downloaded from its github repository, \url{https://github.com/zingale/pyro2} as:
\begin{verbatim}
git clone https://github.com/zingale/pyro2
\end{verbatim}

The structure of the code and descriptions of the various runtime
parameters is found on the pyro webpage,
\url{http://bender.astro.sunysb.edu/hydro_by_example/}.

\section{The \pyro\ Solvers}

\pyro\ offers the following 2-d solvers:
\begin{itemize}
\item {\em advection}: an unsplit, second-order method for linear advection,
  following the ideas from Chapter~\ref{ch:advection}.

\item {\em compressible}: an unsplit, second-order compressible hydrodynamics
  solver using the piecewise linear reconstruction discussed in Chapter~\ref{ch:compressible}.

\item {\em diffusion}: a second-order implicit diffusion solver, based
  on the ideas from Chapter~\ref{ch:diffusion}.

\item {\em incompressible}: a second-order incompressible hydrodynamics
  solver using a cell-centered approximate projection, as discussed
  in Chapter~\ref{ch:incompressible}.

\end{itemize}

\section{\pyro 's Structure}




