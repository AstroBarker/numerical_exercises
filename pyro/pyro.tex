\label{app:pyro}

%% \begin{center}
%% \includegraphics[width=0.125\linewidth]{pyro-sm}
%% \end{center}

\begin{quote}
\noindent {\em Here we describe the basic structure and use of the
pyro code that implements many of the algorithms described in these
notes.}
\end{quote}



\section{Getting \pyro}

\pyro\ can be downloaded from its github repository, \url{https://github.com/zingale/pyro2} as:
\begin{verbatim}
git clone https://github.com/zingale/pyro2
\end{verbatim}

The structure of the code and descriptions of the various runtime
parameters is found on the pyro webpage,
\url{http://zingale.github.io/pyro2/}, and described
in \cite{pyro}.

\pyro\ uses the {\sf matplotlib} and {\sf numpy} libraries.
Several routines are written in Fortran and need to be compiled.  The
script {\tt mk.sh} will build the packages.  This requires that {\tt
f2py} is installed.  


\section{The \pyro\ Solvers}

\pyro\ offers the following 2-d solvers:
\begin{itemize}
\item {\em advection}: an unsplit, second-order method for linear advection,
  following the ideas from Chapter~\ref{ch:advection}.

\item {\em compressible}: an unsplit, second-order compressible hydrodynamics
  solver using the piecewise linear reconstruction discussed in Chapter~\ref{ch:compressible}.

\item {\em diffusion}: a second-order implicit diffusion solver, based
  on the ideas from Chapter~\ref{ch:diffusion}.

\item {\em incompressible}: a second-order incompressible hydrodynamics
  solver using a cell-centered approximate projection, as discussed
  in Chapter~\ref{ch:incompressible}.

\item {\em lm\_atm}: a low-Mach number hydrodynamics solver for
  atmospheric flows, as discussed in S~\ref{sec:lm:atm}
  
\item {\em multigrid}: a multigrid solver for constant-coefficient Helmholtz
  elliptic equations.  This follows the ideas from Chapter~\ref{ch:multigrid},
  and is used by the diffusion and incompressible solvers.

\end{itemize}

\section{\pyro 's Structure}

The grid structure is managed by the {\tt patch.Grid2d} class.  Data
is that lives on the grid is contained in a {\tt
  patch.CellCenterData2d} object.  Methods are available to provide
access to the data and fill the ghost cells.

Each \pyro\ solver is its own python module.  All but the multigrid
solver represent time-dependent problems.  Each of these provide a
{\tt Simulation} class that provides the routines necessary to
initialize the solver, determine the timestep, and evolve the solution
in time.  Each solver has one or more sets of initial conditions
defined in the solver's {\tt problems/} directory.

All time-dependent problems are run through the {\tt pyro.py} script.
The general form is: \\[0.5em]
{\tt ./pyro.py} {\em solver problem inputs} \\[0.5em]
where {\em solver} is one of {\tt advection}, {\tt compressible},
{\tt diffusion}, {\tt incompressible}, {\em problem} is
one of the problems defined in the solver's {\tt problems/}
directory, and {\em inputs} is the name of an input file
that defines the values of runtime parameter. 

The possible runtime parameters and their defaults are defined in the
{\tt \_defaults} files in the main directory and each solver and
problem directory.  Note that the inputs file need not be in the 
{\tt pyro2/} directory.  The solver's {\tt problems/} directory
will also be checked.

\section{Running \pyro}

A simple Gaussian advection simulation is provided by the advection
{\em smooth} problem.  This is run as:
\begin{verbatim}
./pyro.py advection smooth inputs.smooth
\end{verbatim}
As this is run, the solution will be visualized at each step,
showing the progression of the simulation.  


A list of the problems available for each solver is given in
Table~\ref{tab:problems}.  For the multigrid solver, there are scripts
available in the {\tt multigrid/} directory that illustrate its use.



\begin{sidewaystable*}
\centering
\renewcommand{\arraystretch}{1.2}
\begin{tabular}{lp{1in}lp{4.0in}}
\hline
{\bf solver} & {\bf section} & {\bf problem} 
   & {\bf problem description} \\
\hline
%
\multirow{2}{*}{\tt advection}
   & \multirow{2}{*}{\S~\ref{adv:sec:unsplit_2d}}
   & {\tt smooth} & advect a smooth Gaussian profile \\
   & & {\tt tophat} & advect a discontinuous tophat profile \\
\hline
%
\multirow{2}{*}{\tt advection\_rk} 
   & \multirow{2}{*}{\S~\ref{adv:sec:mol_2d}}
   & {\tt smooth} & advect a smooth Gaussian profile \\
   & & {\tt tophat} & advect a discontinuous tophat profile \\
\hline
%
\multirow{7}{*}{\tt compressible} 
   & \multirow{7}{*}{Ch.~\ref{ch:compressible}}
   & {\tt advect} & a Gaussian profile in the density field advected diagonally (in constant pressure background) \\
   & & {\tt bubble} & a buoyant bubble in a stratified atmosphere \\
   & & {\tt kh} & setup a shear layer to drive Kelvin-Helmholtz instabilities \\
   & & {\tt quad} & 2-d Riemann problem based on \cite{schulz-rinne:1993}  \\
   & & {\tt rt}  & a simple Rayleigh-Taylor instability\\
   & & {\tt sedov} & the classic Sedov-Taylor blast wave \\
   & & {\tt sod} & the Sod shock tube problem  \\
\hline
%
{\tt diffusion} & \S~\ref{diff:sec:implicit_mg} & {\tt gaussian} 
   & diffuse an initial Gaussian profile \\
\hline
%
\multirow{2}{*}{\tt incompressible} 
   & \multirow{2}{*}{Ch.~\ref{ch:incompressible}}
   & {\tt converge} & A simple incompressible problem with known
                      analytic solution.\\
   & & {\tt shear} & a doubly-periodic shear layer \\
\hline
%
\multirow{1}{*}{\tt lm\_atm} 
   & \multirow{1}{*}{Ch.~\ref{sec:lm:atm}}
   & {\tt bubble} & a buoyant bubble in a stratified atmosphere \\
\hline
\end{tabular}
\caption{\label{tab:problems} \pyro\ solvers and their distributed problems.  The relevant section in these notes that describes the implementation is also given.}
\renewcommand{\arraystretch}{1.0}
\end{sidewaystable*}




