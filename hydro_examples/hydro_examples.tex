\label{app:hydroex}

\begin{quote}
\noindent {\em Here we describe the basic structure and use of the
\hydroex\ codes that implement standalone, simple 1-d versions of the algorithms described in these
notes.}
\end{quote}

The \hydroex\ codes are simple 1-d solvers written in python that illustrate many of the
ideas in these nodes.  They are used for making many of the figures found throughout
(wherever the ``\hydroexdoit{}'' note is found).

\section{Getting \hydroex}

The \hydroex\ codes are hosted on github:\\
\url{https://github.com/zingale/hydro_examples}

There are a few ways to get them.  The simplest way is to just clone
from github on the commandline:

\begin{verbatim}
git clone https://github.com/zingale/hydro_examples
\end{verbatim}

This will create a local git repo on your machine with all of the
code.  You can then run the scripts, ``as is'' without needing to go
to github anymore.  Periodically, there may be updates to the scripts,
which you can obtain by issuing ``{\tt git pull}'' in the {\tt
hydro\_examples/} directory.  Note that if you have made any local
changes to the scripts, \git\ may complain.  The process of merging
changes is descibed online in various places,
e.g. ``\href{https://help.github.com/articles/resolving-a-merge-conflict-from-the-command-line}{Resolving
a merge conflict from the command line}'' from github.

Alternately, you can use github's web interface to fork it.  Logon to
github (or create an account) and click on the ``Fork'' icon on the
\hydroex\ page.  You can then interact with your version of the repo
as needed.  This method will allow you to push changes back to github,
and, if you think they should be included in the main \hydroex\ repo,
issue a pull-request.

\section{\hydroex\ codes}

The codes in \hydroex\ are organized into directories named after the
chapters in these notes.  Each directory is self-contained.  The
following scripts are available:

\begin{itemize}

\item {\tt advection/}
  
  \begin{itemize}
  \item {\tt advection.py}: a 1-d second-order linear advection solver
    with a wide range of limiters.

  \item {\tt fdadvect\_implicit.py}: a 1-d first-order implicit
    finite-difference advection solver using periodic boundary
    conditions.

  \item {\tt fdadvect.py}: a 1-d first-order explicit
    finite-difference linear advection solver using upwinded
    differencing.

  \end{itemize}

\item {\tt burgers/}

  \begin{itemize}
  \item {\tt burgers.py}: a 1-d second-order solver for the inviscid
    Burgers' equation, with initial conditions corresponding to a
    shock and a rarefaction.

  \end{itemize}

\item {\tt compressible/}

  \begin{itemize}
  \item {\tt riemann-phase.py}: draw the Hugoniot curves in the $u$-$p$ plane
     for a pair of states that comprise a Riemann problem.
  \end{itemize}

\item {\tt multigrid/}

  \begin{itemize}
  \item {\tt mg\_converge.py}: a convergence test of the multigrid
    solver.  A Poisson problem is solved at various resolutions and
    compared to the exact solution.  This demonstrates second-order
    accuracy.

  \item {\tt mg\_test.py}: a simple driver for the multigrid solver.
    This sets up and solves a Poisson problem and plots the behavior 
    of the solution as a function of V-cycle number.

  \item{\tt multigrid.py}: a multigrid class for cell-centered data.
    This implements pure V-cycles.  A square domain with $2^N$ zones
    ($N$ a positive integer) is required.

  \item {\tt patch1d.py}: a class for 1-d cell-centered data that
    lives on a grid.  This manages the data, handles boundary
    conditions, and provides routines for prolongation and restriction
    to other grids.

  \end{itemize}

\item {\tt diffusion/}

  \begin{itemize}
    \item {\tt diffusion-explicit.py}: solve the constant-diffusivity
      diffusion equation explicitly.  The method is first-order
      accurate in time, but second-order in space.  A Gaussian profile
      is diffused---the analytic solution is also a Gaussian.

    \item {\tt diffusion-implicit.py}: solve the constant-diffusivity
      diffusion equation implicitly.  Crank-Nicolson time-discretization
      is used, resulting in a second-order method.  A Gaussian profile
      is diffused.

  \end{itemize}
  
\item {\tt multiphysics/}

  \begin{itemize}
  \item {\tt burgersvisc.py}: solve the viscous Burgers equation.
     The advective terms are treated explicitly with a second-order
     accurate method.  The diffusive term is solved using an 
     implicit Crank-Nicolson discretization.  The overall coupling
     is second-order accurate.

  \item {\tt diffusion-reaction.py}: solve a diffusion-reaction
    equation that propagates a diffusive reacting front (flame).  A
    simple reaction term is modeled.  The diffusion is solved using
    a second-order Crank-Nicolson discretization.  The reactions
    are evolved using the VODE ODE solver (via SciPy).  The two
    processes are coupled together using Strang-splitting to be
    second-order accurate in time.
     

  \end{itemize}

\end{itemize}


