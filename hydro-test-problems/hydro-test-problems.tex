\section{Testing}

There are a large number of standard hydrodynamics test problems that
should be run on any implementation of an algorithm.  These problems
uncover strengths and weaknesses in your choice of algorithm and also
simple coding bugs.

An ideal test problem has an analytic solution that can be compared to
directly.

Convergence of your results is also an important test.

Recall that for a finite-volume discretization, to second-order, the
value of a function evaluated at the cell-center is the same as the
average of that function over the zone.  Usually this means that you
can simply initialize your test problem using cell-center coordinates.
But for problems that are not well aligned with your grid, e.g., a
spherical initial function mapped onto your Cartesian grid, it is
beneficial to try to initialize the average value in the zone.  A
common way to do this is to sub-divide a zone into a number of
sub-zones, initialize each of the sub-zones, and then average the
sub-zones back to the original zone.  \MarginPar{show figure}


\section{Shock tube problems}



\section{Advection}


\section{Sedov}



\section{Gresho vortex}



